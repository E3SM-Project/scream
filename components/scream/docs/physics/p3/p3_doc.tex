\section{Predicted Particle Properties (P3)}

Describe scheme in general (copy/paste Hassan's existing doc)

%================================
\subsection{Autoconversion}
%================================

Autoconversion is the process in which cloud dropelts collide and coalesce to form rain drops. 

\subsubsection{Theory}

The current implementation of P3 has three options for paramaterizing autoconversion are those described 
in Seifert and Beheng (2001), Beheng (2004), and Khroutdinov and Kogan (2000). The default is 
Khroutdinov and Kogan (2000) which calculates the rain mass tendency as: 

\frac{\partial q_{c}}{\partial t} = cq_{c}^{a}N_{c}^{b}

Where a,b, and c are fitting parameters. A best fit based on the data from the ACE-1 field 
campaign and an explicit microphysics model results in a = 2.47, b = - 1.79, and c = 1350.

The rain number tendency is calculated based on the assumption that newly formed rain drops 
have a diameter of 25 um and can thus be calculated as:

\frac{\partial N_{c}}{\partial t} = \frac{\frac{\partial N_{c}}{\partial t}}{\frac{4pi\rho _{w}}{3\rho _{a}}r_{o}^3}

\subsubsection{Numerical Methods}

The above tendencies are calculated discretely by setting the differential dt equal to the model physics timestep.

\subsubsection{Computational Implementation}

None.

\subsubsection{Verification}

Describe testing strategy

%================================
\subsection{Accretion}
%================================

Say what accretion does

\subsubsection{Theory}

Put explanation of continuous equations here.

\subsubsection{Numerical Methods}

Describe the numerical methods used to discretize the equations here.

\subsubsection{Computational Implementation}

Describe the strategies used (if any) to ensure good computational performance.

\subsubsection{Verification}

Describe testing strategy

%================================
%... and so on...
%================================

